\newcommand{\hdr}[1]{
    \multicolumn{4}{c}{~} \\
    \multicolumn{4}{c}{\textbf{#1}} \\
    \hline
}

\beginlongtable{ | l | l | l | X | }
\hline
\textbf{Name} & \textbf{P} & \textbf{Types} & \textbf{Behavior} \\
\hline
\hline
\endhead

\hdr{Meta}
Function & 1 & O\ra c & Result is a function (associated with Null) with the first argument as its body \\
\hline

\hdr{Objects}
This & 0 & & Result is the value of the This register, which is initially the
value called. \\
\hline
ThisSet & 1 & & The This register is set to the value of argument 0. Result is
the value. \\
\hline
Arg & 0 & & Result is the value of the Argument register, which is initially
the second argument of the Call expression used to call this function. \\
\hline
Null & 0 & & Evaluates to Null, with no primitive associated. \\
\hline
Global & 0 & & Evaluates to Global, with no primitive associated. \\
\hline
ExpandGlobal & 1 & i & Replaces Global with a new object, the size of which is
the original size of Global plus the supplied integer argument. Evaluates to
the old size of Global. \\
\hline
Version & 0 & & Evaluates to Version, with no primitive associated. \\
\hline
New & 1 & i & A new object, the size of which is specified by the first
argument, is created. Each field of the new object has the value of Null with
no primitive associated. The result is the new object, with no primitive
associated. \\
\hline
Length & 1 & \ra i & Given an object, evaluates to its size, with Null
associated. \\
\hline
Member & 2 & oi & Given an object and an index, evaluates to the value at that
index of the object. \\
\hline
MemberSet & 3 & oio & Given a ``to'' object, an index, and a ``from'' value,
sets the value of the ``to'' object at that index to the ``from'' value. Result
is the ``from'' value. \\
\hline
Equal & 2 & \ra i & Given two objects, if they are referentially identical
(ignoring the primitive), results in the integer 1 associated with Null.
Otherwise results in the integer 0 associated with Null. \\
\hline
Object & * & & Creates a new object, the size of which is the number of
arguments. The values of the fields are the values of the arguments, in the
order they appear. Result is the new object, with no primitive associated. \\
\hline
Concat & 2 & & Creates a new object, the size of which is the sum of the size
of the two argument objects. Given objects of size N and M, the values of the
new object's fields $[0..N-1]$ are set to the values of the first argument
object's fields, and the values of the new object's fields $[N..N+M-1]$ are set
to values of the second argument object's fields. Result is the new object,
with no primitive associated. \\
\hline
ConcatT & 2 & & See the Fythe transform engine's documentation. \\
\hline
Slice & 3 & oii\ra a & Given an object and two indexes X and Y, creates a new
object of size $Y-X$. The values of the new object's fields $[0..Y-X]$ are set
to the values of the argument object's fields $[X..Y-1]$. Result is the new
object, with no primitive associated. \\
\hline

\hdr{Temporaries\footnote{Fythe provides infinite temporaries, but the mapping of temporaries to registers or similar is implementation-defined.}}
Temp & 1 & i & See the Temporaries section. \\ % FIXME: make one :)
\hline
TempSet & 2 & io &\\
\hline

\hdr{Type detection\footnote{These functions are purely heuristic. Although
they must evaluate to 1 for any correctly-typed value, they may also evaluate
to 1 for other values, although the result of using these values as the
indicated type is undefined. That is, any implementation is allowed to produce
incorrect results in the form of false negatives if, for instance, it cannot
distinguish between types; proper type management is the responsibility of the
programmer or language implementer, not Fythe.}}
ValidString & 1 & \ra i & \validshpiel{a string} \\
\hline
ValidFunction & 1 & \ra i & \validshpiel{a function} \\
\hline
ValidInteger & 1 & \ra i & \validshpiel{an integer} \\
\hline
ValidFloat & 1 & \ra i & \validshpiel{a floating-point number} \\
\hline
ValidTable & 2 & oi\ra i & Will result in the integer 1 or 0, in either case
associated with Null. Given an object and an index, if the value of the indexed
field of the object was created as a table, the result will be 1, but the
result 0 does not indicate that the value was not created as a table, only that
the VM can distinguish the value from a table. A VM should not return 1 if
using the value as a table will result in the VM crashing. \\
\hline
ValidThread & 1 & \ra i & \validshpiel{a thread identifier} \\
\hline
ValidLock & 1 & \ra i & \validshpiel{a lock} \\
\hline

\hdr{Function and call-stack}
Call & 2 & co & Call the given function. The result is the result of the
function's body, when evaluated with the value of the This register being the
function value (that is, the first argument), and the value of the Arg register
being the second argument. \\
\hline
Throw & 1 & & See the Exceptions section. \\ % FIXME: write one :)
\hline
Catch & 2 & OO &\\
\hline
Caught & 0 & &\\
\hline

\hdr{Behavioral}
If & 3 & iOO & The first argument is evaluated. If its value is 0, then the
third argument is evaluated and its result is the result of this If. Otherwise,
the second argument is evaluated and its result is the result of this If. \\
\hline
While & 2 & IO & In a loop:
\begin{itemize}
\item The first argument is evaluated.
\item If its value is 0, the loop is terminated and the result of this While is
the result of the most recent evaluation of the second argument; if the second
argument has not yet been evaluated, the result of this While is Null with no
primitive associated.
\item The second argument is evaluated, and its result remembered.
\end{itemize} \\
\hline
Seq & * & & Each argument is evaluated in sequence. The result is the value of
the last argument. (Seq must have at least one argument) \\
\hline

\hdr{Primitives}
Associate & 2 & po\ra p & Results in the primitive value of the first argument
associated with the object of the second argument. \\
\hline
Dissociate & 1 & p\ra o & Results in the object of the first argument with no
primitive associated. Note that the equivalent for getting a dissociated
primitive value is to associate it with Null. \\
\hline

\hdr{Strings}
SConcat & 2 & ss\ra s & Given two strings, results in a new string which is the
concatenation of the argument strings, associated with Null. \\
\hline
SLength & 1 & s\ra i & Results in the length of the argument string as an
integer, associated with Null. \\
\hline
SSlice & 3 & sii\ra s & Similar to Slice, but slicing over bytes in a string
instead of fields in an object. \\ % FIXME: this description sucks
\hline
SEqual & 2 & ss\ra i & Results in the integer value 1 if the two strings are
equal, 0 otherwise. In either case the value is associated with Null. \\
\hline
SToInteger & 1 & s\ra i & Result is the integer value of the string argument,
interpreted as a decimal ASCII number, associated with Null. \\
\hline
SToIntegerT & 1 & s\ra 1 & See the Fythe transform engine's documentation. \\
\hline

\hdr{Integers}
IWidth & 0 & \ra i & Result is the width, in bits, of a traditional signed
2s-compliment which integers provided by the VM are capable of representing.
For instance, if the VM actually uses 2s-compliment integers, the result is
simply the width of integers it uses. If the VM uses something else (e.g.
floating-point numbers), the result is a conservative estimate of their usable
width as 2s-compliment integers. The result is associated with Null. The
behavior of integers when they do not overflow the size of such an integer is
guaranteed to be correct. \\
\hline
IMul & 2 & ii\ra i & Result is the integer values of the two arguments
multiplied together. If this overflows the VM's representation of integers, the
result is implementation-defined. The result is associated with Null. \\
\hline
IDiv & 2 & ii\ra i & Result is the integer value of the first arguments divided
by the integer value of the second argument, truncated. The result is
associated with Null. \\
\hline
IMod & 2 & ii\ra i & Result is remainder after dividing the integer value of
the first arguments by the integer value of the second argument. If either
argument is negative, the integer result is implementation-defined. The result
is associated with Null. \\
\hline
IAdd & 2 & ii\ra i & Result is the sum of the integer values of the arguments.
If this overflows the VM's representation of integers, the result is
implementation-defined. The result is associated with Null. \\
\hline
ISub & 2 & ii\ra i & Result is the integer value of the second argument
subtracted from the integer value of the first argument. If this overflows the
VM's representation of integers, the result is implementation-defined. The
result is associated with Null. \\
\hline
ISl & 2 & ii\ra i & Result is the integer value of the first argument times
(two to the power of the integer value of the second argument), decimal
truncated. If this overflows the VM's representation of integers, the result is
implementation-defined. The result is associated with Null. \\
\hline
ISr & 2 & ii\ra i & Result is the integer value of the first argument times
(two to the power of the negated integer value of the second argument), decimal
truncated. If this overflows the VM's representation of integers, the result is
implementation-defined. The result is associated with Null. \\
\hline
INot & 1 & i\ra i & If the argument's integer value is 0, the result is the
integer value 1. Otherwise the result is the integer value 0. In either case,
the result is associated with Null. \\
\hline
IOr & 2 & ii\ra i & If the integer values of both arguments are 0, the result
is the integer value 0. Otherwise the result is the integer value 1. In either
case, the result is associated with Null. \\
\hline
IAnd & 2 & ii\ra i & If the integer values of both arguments are 1, the result
is the integer value 1. Otherwise the result is the integer value 0. In either
case, the result is associated with Null. \\
\hline
IByte & 1 & i\ra s & The result is a string, of length one, the value of the
first byte of which is the integer value of the first argument. If the first
argument has a value less than 0 or greater than 255, then the result is
implementation-defined. The result is associated with Null. \\
\hline
IEqual & 2 & ii\ra i & If the integer values of both arguments are equal, the
result is the integer value 1. Otherwise the result is the integer value 0. In
either case, the result is associated with Null. \\
\hline
INe & 2 & ii\ra i & If the integer values of both arguments are not equal, the
result is the integer value 1. Otherwise the result is the integer value 0. In
either case, the result is associated with Null. \\
\hline
ILt & 2 & ii\ra i & If the integer value of the first argument is strictly less
than the integer value of the second argument, the result is the integer value
1. Otherwise the result is the integer value 0. In either case, the result is
associated with Null. \\
\hline
ILte & 2 & ii\ra i & If the integer value of the first argument is strictly
less than the integer value of the second argument, the result is the integer
value 1. Otherwise the result is the integer value 0. In either case, the
result is associated with Null. \\
\hline
IGt & 2 & ii\ra i & If the integer value of the first argument is strictly less
than the integer value of the second argument, the result is the integer value
1. Otherwise the result is the integer value 0. In either case, the result is
associated with Null. \\
\hline
IGte & 2 & ii\ra i & If the integer value of the first argument is strictly
less than the integer value of the second argument, the result is the integer
value 1. Otherwise the result is the integer value 0. In either case, the
result is associated with Null. \\
\hline
IToFloat & 1 & i\ra f & \\
\hline

\hdr{Floats}
FMul & 2 & ff\ra f &\\
\hline
FDiv & 2 & ff\ra f &\\
\hline
FMod & 2 & ff\ra f &\\
\hline
FAdd & 2 & ff\ra f &\\
\hline
FSub & 2 & ff\ra f &\\
\hline
FEqual & 2 & ff\ra i &\\
\hline
FNe & 2 & ff\ra i &\\
\hline
FLt & 2 & ff\ra i &\\
\hline
FLte & 2 & ff\ra i &\\
\hline
FGt & 2 & ff\ra i &\\
\hline
FGte & 2 & ff\ra i &\\
\hline
FToInteger & 1 & f\ra i &\\
\hline

\hdr{Tables}
MInit & 2 & oi & (always returns (Null))\\
\hline
MGet & 3 & ois & (table must be in an array index, uninitialized index is fine)\\
\hline
MSet & 4 & oiso &\\
\hline
MList & 2 & oi &\\
\hline
MContains & 3 & ois\ra i &\\
\hline

\hdr{Threading}
Spawn & 2 & co\ra t &\\
\hline
Join & 1 & t & (FIXME: what's the return?)\\
\hline
Yield & 0 & & (always returns (Null))\\
\hline
LMutexNew & 0 & \ra l &\\
\hline
LMutexLock & 1 & l\ra l &\\
\hline
LMutexTryLock & 1 & l\ra i &\\
\hline
LMutexUnlock & 1 & l\ra l &\\
\hline
LMutexDestroy & 1 & l\ra o & (alwas returns (Null))\\
\hline
LSemNew & 0 & \ra l &\\
\hline
LSemPost & 1 & l\ra l &\\
\hline
LSemWait & 1 & l\ra l &\\
\hline
LSemDestroy & 1 & l\ra o & (alwas returns (Null))\\
\hline
LRWNew & 0 & \ra l &\\
\hline
LRWRLock & 1 & l\ra l &\\
\hline
LRWRTryLock & 1 & l\ra i &\\
\hline
LRWRUnlock & 1 & l\ra l &\\
\hline
LRWRLock & 1 & l\ra l &\\
\hline
LRWRTryLock & 1 & l\ra i &\\
\hline
LRWRUnlock & 1 & l\ra l &\\
\hline
LRWDestroy & 1 & l\ra o & (always returns (Null))\\
\hline
Cas & 4 & oioo\ra i & (only compares and swaps the object, not the primitive)\\
\hline
PCas & 4 & oipp\ra i & (only compares and swaps the primitive, not the object)\\
\hline

\hdr{Runtime}
Include & 1 & s\ra s & (or a throw for failure)\\
\hline
Parse & 2 & ss &\\
\hline
Transform & 1 & & (should always return code, but this is a property of the transforms and so not guaranteed)\\
\hline

\hdr{Grammar}
GAdd & 3 & soo\ra s &\\
\hline
GRem & 1 & s\ra s &\\
\hline
GCommit & 0 & & (always returns Null)\\
\hline

\hdr{Transform}
TAddPhase & 2 & ss\ra i &\\
\hline
TAddTree & 3 & soo\ra i &\\
\hline
TAddFunction & 3 & soc\ra i &\\
\hline
TRem & 1 & s\ra i &\\
\hline

\eendlongtable
