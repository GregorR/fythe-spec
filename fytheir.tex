\chapter{Fythe Objects and IR}

Every IR expression is reduced at runtime to a value. A value in Fythe is an
object reference associated optionally with a primitive value. A primitive
value is a string, function, integer, floating-point number, or
implementation-specific primitive. All primitives are immutable. An object is
an array of values; that is, a set of values indexed by keys which range from 0
to the size of the object, exclusively. There are several special objects,
notably Null, which is a particular object of size 0 used as a nonce.

The IR is a tree, formed by Fythe values. Although there is no canonical syntax
for Fythe values, this specification uses LISP-like list expressions,
conforming to the following BNF grammar (using regular expressions as
terminals):

% FIXME: I'd like to do this in a less Ad-Hoc way, but I know of none
\begin{verbatim}
WhiteN      = /[ \t\r\n]*/

Expression  = /\(/ WhiteN List /\)/ WhiteN
            | /\(/ WhiteN /\)/ WhiteN
            | /[^\(\)\{\}0-9 \t\r\n][^\(\)\{\} \t\r\n]*/ WhiteN
            | /[0-9]+/ WhiteN

List        = Expression
            | Expression List
\end{verbatim}

An expression of the first two forms represents an object associated with no
primitive. An expression of the third form represents a string associated with
Null. An expression of the fourth form represents an integer associated with
Null. This syntax has no means of expressing objects other than Null associated
with primitive values, as these have no behavior in the Fythe IR.

A Fythe IR expression is an object in which the first element is a string
literal, the value of which is the type of expression, and the remaining
elements are arguments. For instance,
\texttt{(Null)}
is a Fythe IR expression which (at runtime) reduces to the value Null
(associated with no primitive),
\texttt{(New 2)}
is a Fythe IR expression which reduces to a newly-created object of size 2, and
\texttt{(Concat (New 2) (New 2))}
is a Fythe IR expression which (rather pointlessly) reduces to the
concatenation of two newly-produced objects of size 2, to produce a new object
of size 4.

The following table is a list of every Fythe IR expression, its parameter
count, the types its arguments must have (when reduced at runtime), and any
additional notes. The types are expressed as a string of letters, with each
letter corresponding to one argument or return type. Where the input types are
all generic (just objects, no primitives necessary) the list is elided, as is
the return type when it may be any object or primitive type. The association
between letters and types is in Figure~\ref{irtypes}. Not all expressions are
yet documented.

\LTXtable{\linewidth}{irexpressions.tex}

\begin{wrapfigure}{L}{0.35\textwidth}
\LTXtable{\linewidth}{irtypes.tex}

A capitalized letter indicates that the relevant argument may be evaluated a
number of times other than one. That is, the argument will be evaluated
conditionally, multiple times, or deferred.

\caption{Fythe type identifiers}
\label{irtypes}
\end{wrapfigure}
