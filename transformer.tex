\chapter{The Transform and Macro Engine}\label{chap:transformer}

The Fythe parser is free to produce any tree of objects, not only those
conforming to the Fythe IR. This is so that concerns such as binding,
optimization and analysis may be offloaded to separate phases, in the transform
and macro engine.

Fundamentally, the transform engine works by running transforms in a number of
\textit{phases} in order. Each phase has a number of \textit{transforms}. Each
transform consists of a \textit{pattern} to match in the input tree, and either
an output tree, or a function to produce an output tree. New phases are added
by specifying a predecessor phase; i.e., a new phase is always inserted
immediately after an existing phase. Transforms are added to phases in the
order they are encountered in code, and executed in the reverse of that order.

At least one phase exists before user code executes, with the name
\texttt{Parsing}. This phase contains no transforms, and is always the earliest
phase. Parsing is not actually performed by the ``Parsing'' phase, the name is
chosen simply to represent the beginning of transformation, which is the end of
parsing. An implementation may choose to include additional phases, which
should, by convention, have names beginning with \texttt{Impl.}.
