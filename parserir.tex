\beginlongtable{ | l | l | l | X | }
\hline
\textbf{Name} & \textbf{P} & \textbf{Types} & \textbf{Behavior} \\
\hline
\hline
\endhead

\hline
Parse & 5 & siiss & The input specified by the fifth argument is parsed by the
parser named by the fourth argument. If the parsing succeeds, the result is an
object with the following members (respectively): the value generated by
parsing, the remaining text associated with Null, the line number if the first
character not parsed associated with Null, and the column of the first
character not parsed associated with Null.  If parsing fails, the result is
Null. The first three arguments are a filename, line number and column number,
and are only present to allow Parse to create useful error messages. \\
\hline
GAdd & 3 & soo\ra s & An expression and target is added to the nonterminal
named by the first argument. The second argument is the expression to parse,
and must be an object, each element of which is a string naming a parser. The
third argument is the target. The result is the first argument. \\
\hline
GRem & 1 & s\ra s & All expressions and targets associated with the parsed
named by the first argument are removed from the grammar. The result is the
first argument. \\
\hline

\eendlongtable
